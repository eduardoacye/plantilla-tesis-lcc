\section{Introducción}

En este capítulo se abordará la manera agregar citas de la bibliografía; referencias a figuras y 
cuadros; algoritmos.\\

\section{Citas textuales y referencias}

En el documento se puede hacer referencia tanto a \emph{labels} (etiquetas) o entradas en la 
bibliografía.\\

\subsection{A figuras y cuadros}
En el caso de las referencias a figuras o cuadros se utiliza el comando \texttt{\textbackslash 
cref\{label\}}, en donde \texttt{label} es el nombre del identificador de la figura o tabla. Por 
ejemplo, en el \cref{Ta:primer ejemplo2} podemos observar que se ven unas fórmulas matemáticas 
chistosas.\\

\subsection{A la bibliografía}
En el caso de las referencias a entradas de la bibliografía se utiliza el comando 
\texttt{\textbackslash cite\{label\}}, en donde \texttt{label} es el nombre de la entrada de la 
referencia bibliográfica. Por ejemplo, el artículo de Sachdeva \cite{chackra} presenta una 
novedosa y arcana técnica para la regularización de los \emph{chakras}.\\

\subsection{Múltiples referencias}
En ocaciones es deseable referenciar a dos o mas etiquetas o entradas de la bibliografía, en dicho 
caso se pueden enlistar los argumentos de los comandos \texttt{\textbackslash cref} y 
\texttt{\textbackslash cite} separados por comas. Por ejemplo, dos trabajos de tesis que fueron 
inspiración para la elaboración de las \cref{fig:pdf,fig:png} fueron 
\cite{ejemplo_maestria,ejemplo_tesis_doctoral}.

\section{Códigos}

\subsection{Seudocódigos}
Para escribir algoritmos en seudocódigo con palabras clave en español podemos utilizar el entorno 
\texttt{algorithm} y \texttt{algorithmic}, el primero nos permite etiquetar y titular el 
algoritmo, mientras que el segundo es el entorno en donde se escribe el seudocódigo. Por ejemplo, 
el algoritmo \cref{alg1} describe la operación de potenciación.\\

\begin{algorithm}
	\caption{Calcula $y = x^n$}
    \label{alg1}
    \begin{algorithmic}
		\REQUIRE $n \geq 0 \vee x \neq 0$
        \ENSURE $y = x^n$
        
        \STATE $y \leftarrow 1$
        
        \IF{$n < 0$}
        	\STATE $X \leftarrow 1 / x$
            \STATE $N \leftarrow -n$
        \ELSE
        	\STATE $X \leftarrow x$
            \STATE $N \leftarrow n$
        \ENDIF
        
        \WHILE{$N \neq 0$}
        	\IF{$N$ es par}
            	\STATE $X \leftarrow X \times X$
                \STATE $N \leftarrow N / 2$
            \ELSE[$N$ es impar]
            	\STATE $y \leftarrow y \times X$
                \STATE $N \leftarrow N - 1$
            \ENDIF
        \ENDWHILE
	\end{algorithmic}
\end{algorithm}

Podemos tomar el ejemplo del algoritmo \emph{quicksort} del Cormen, el cual tiene un tiempo de 
ejecución en el peor de los casos de $\Theta \left(n^2 \right)$ y en el caso promedio $\Theta 
\left( n \lg n \right)$ sobre un arreglo de entrada de $n$ números y escribirlo de manera similar 
a como aparece en el libro \cref{alg:quicksort}.\\

\begin{algorithm}
	\caption{Ordenamiento \emph{quicksort}}
    \label{alg:quicksort}

	\vspace{10 pt}
	$\textsc{Quicksort}\left(A,p,r\right)$

    \begin{algorithmic}[1]
		\IF{$p < r$}
        	\STATE $q \leftarrow \textsc{Partition} \left( A, p, r \right)$
            \STATE $\textsc{Quicksort}\left( A, p, q-1 \right)$
            \STATE $\textsc{Quicksort}\left( A, q+1, r \right)$
        \ENDIF
	\end{algorithmic}
    
    \vspace{10 pt}
    $\textsc{Partition}\left(A,p,r\right)$
    \begin{algorithmic}[1]
    	\STATE $x \leftarrow A\left[r\right]$
        \STATE $i \leftarrow p-1$
        \FOR{$j\leftarrow p$ \TO $r-1$}
        	\IF{$A\left[j\right] \leq x$}
            	\STATE $i \leftarrow i+1$
                \STATE intercambia $A\left[i\right]$ con $A\left[j\right]$
            \ENDIF
        \ENDFOR
        \STATE intercambia $A\left[i+1\right]$ con $A\left[r\right]$
        \RETURN $i + 1$
    \end{algorithmic}
\end{algorithm}

\subsection{Códigos fuente}

Si el código que deseamos mostrar está escrito en algún lenguaje de programación en particular, 
utilizamos los entornos \texttt{listing} y \texttt{minted}. Por ejemplo, el 
\cref{code:rubyexample} es un ejemplo de código en el lenguaje \emph{Ruby}.\\

\begin{listing}[H]
\begin{minted}[frame=single]{ruby}
lazy_integers = (1..Float::INFINITY).lazy
lazy_integers.collect { |x| x ** 2 }.
              select { |x| x.even? }.
              reject { |x| x < 1000 }.
              first(5)
\end{minted}
\caption{Ejemplo de código en Ruby}
\label{code:rubyexample}
\end{listing}

El programa descrito en el \cref{code:helloworld} es todo un clásico en el mundo de la 
programación.\\

\begin{listing}[H]
\begin{minted}[frame=single]{c}
#include <stdio.h>

main()
{
    printf("hello, world\n");
}
\end{minted}
\caption{Clásigo \emph{hola mundo} en el lenguaje \emph{C}.}
\label{code:helloworld}
\end{listing}
