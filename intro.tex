\xsection{Descripción del trabajo}

En la introducción es muy importante dejar claro cual era la
problemática a tratar, la justificación de dicho trabajo, los
objetivos (general y específicos) así como las metas. También es
importante acotar los alcances del trabajo. Esta información ya se
tiene del proyecto de tesis que se presentó antes de la realización de
la tesis, aunque en muchos casos es necesario adaptarla a lo que se
hizo realmente.

En un trabajo de tesis no es extraño empezar con un objetivo inicial y
conforme se desarrolla el trabajo derivar a otro problema o acotar el
problema original. Si este es el caso, es conveniente dejar claro en
la introducción de la tesis los objetivos, metas, justificación y
descripción \emph{del trabajo que se está presentando}.

Este capítulo no es muy largo y se escribe más fácilmente después de
escribir los capítulos del cuerpo de la tesis. Las secciones
siguientes (aportaciones principales y organización del trabajo)
suelen ser muy cortas (un párrafo o dos).

\xsection{Aportaciones principales}

Es una buena idea dejar en claro cuales son las aportaciones del
trabajo desde el principio. En este apartado se puede especificar
cuales son los trabajos, programas, publicaciones, prototipos o
revisiones realizadas. Así, se facilita la tarea del comité revisor y
facilita la lectura para quienes estén interesados en leer el trabajo.

Recuerda que es una tesis de licenciatura, y el hecho de estudiar un
artículo (o una técnica no vista durante la carrera), desmenuzarlo,
entenderlo, explicarlo y/o programarlo es suficiente. En una tesis de
licenciatra no se espera que se realice una aportación original.

\xsection{Organización del trabajo}

Facilita la lectura del trabajo describir que es lo que se presenta en el capitulo uno,
capitulo dos, etc. así como cuales son los apéndices que se agregan
al trabajo de tesis.
