\section{¿Porque poner apéndices?}

Hay cosas en una tesis que costaron mucho trabajo, que son
interesantes de mostrar para simplificar el trabajo de otras
personas que continúen con el trabajo realizado, pero que son áridos
de leer o no aportan mucho en el desarrollo del tema principal. En
ese caso los apéndices es la solución ideal para mostrarlos.

Este es el caso de los programas (código fuente) o bases matemáticas
de alguna técnica, en donde la aplicación es lo importante en el
trabajo de tesis. Por favor, incluye código fuente solo si es
extrictamente necesario, o si por la naturaleza del trabajo no se
puede
explicar con pseudocódigo. Es una tesis en Ciencias de la Computación
y se asume que todo el que la lee sabe programar. Un caso particular
donde el código es importante es cuando el trabajo se trata del uso de
una plataforma o una técnica muy específica (como podría ser
\emph{Jess}, o el uso de librerías para \emph{MPI}).

\section{Otros casos}

Otros casos importantes es donde es necesario poner a disposición la
información utilizada de otra fuente, tal como las bases de datos o
las bibliotecas de funciones realizadas por otras personas. Es común
en algunas tesis de agregar un último apéndices con los artículos
que se desarrollaron durante la tesis.
